\newcommand{\diameter}{20}
\newcommand{\xone}{-15}
\newcommand{\xtwo}{160}
\newcommand{\yone}{15}
\newcommand{\ytwo}{-253}

\begin{titlepage}

\begin{tikzpicture}[overlay]
\draw[color=gray]
		 (\xone mm, \yone mm)
  -- (\xtwo mm, \yone mm)
 arc (90:0:\diameter pt)
  -- (\xtwo mm + \diameter pt , \ytwo mm)
	-- (\xone mm + \diameter pt , \ytwo mm)
 arc (270:180:\diameter pt)
	-- (\xone mm, \yone mm);
\end{tikzpicture}


\begin{textblock}{10}[0,0](4,2.5)
  \includegraphics[width=.3\textwidth]{logos/KITLogo.pdf}
\end{textblock}
\changefont{phv}{m}{n}
\vspace*{3cm}
\begin{center}
  \LARGE{Skript zur Stammvorlesung}
  \vspace*{1.5cm}\\
  \Huge{Sicherheit}\\
  \vspace*{3cm}
  \Large{\textbf{Karlsruher Institut für Technologie}\\
  \vspace*{6mm}
  Fakultät für Informatik\\
  \vspace*{4mm}
  Institut für Theoretische Informatik\\
  Arbeitsgruppe für Kryptographie und Sicherheit}\\
  
  \vspace*{2cm}
  Die aktuelle Version des Skriptes befindet sich noch im Aufbau, daher kann weder für Vollständigkeit noch Korrektheit garantiert werden. Hinweise zu Fehlern, Kritik und Verbesserungsvorschläge nehmen wir per Mail an \url{skript-sicherheit@ira.uka.de} entgegen.
  
  \vspace*{2cm}
  Letzte Änderung: \today
\end{center}


\begin{textblock}{10}[0,0](4,16.8)
\tiny{
  \iflanguage{english}
  {KIT -- University of the State of Baden-Wuerttemberg and National Laboratory of the Helmholtz Association}
  {KIT -- Universität des Landes Baden-Würtemberg und nationales Forschungszentrum der Helmholtz-Gesellschaft}
}
\end{textblock}

\begin{textblock}{10}[0,0](14,16.75)
\large{
  \textbf{www.kit.edu}
}
\end{textblock}

\end{titlepage}

\thispagestyle{empty}
\ \vfill
\begin{flushleft}
  Copyright $\copyright$ ITI und Verfasser 2014\\
  \ \\
  Karlsruher Institut für Technologie\\
  Institut für Theoretische Informatik\\
  Arbeitsgruppe für Kryptographie und Sicherheit\\
  Am Fasanengarten 5\\
  76131 Karlsruhe
\end{flushleft}
\newpage

%%% Local Variables:
%%% mode: latex
%%% TeX-master: "skript"
%%% End:
