\chapter{Benutzerauthentifikation}\label{cha11}
\index{Benutzerauthentifikation}
In den vorherigen beiden Kapiteln haben wir betrachtet, wie sich ein
Prover gegenüber einem Verifier identifizieren kann.  Dabei konnten wir
durchaus beachtliche Resultate vorweisen.  Leider kommen die bisher
betrachteten Protokolle nur für die computergestützte Identifizierung
des Provers gegenüber dem Verifier in Frage, denn kaum ein Mensch wird
sich einen komplizierten geheimen Schlüssel für ein Signaturverfahren
merken, geschweige denn den Signaturalgorithmus von Hand ausführen
wollen. Man stelle sich dies im Fall von RSA-basierten Signaturen vor:
Alleine der geheime Schlüssel wird eine für 2048-Bit RSA über 600
Stellen lange Zahl sein.  Auch das Protokoll auf Basis der
Graphdreifärbbarkeit ist nur mühsam von Hand auszuführen, da das
Protokoll oft genug wiederholt werden muss, um echte Sicherheit zu
bieten.

Aus diesem Grund wollen wir uns in diesem Kapitel damit
auseinandersetzen, wie sich Menschen authentifizieren (können), und wie
man eine solche Authentifikation möglichst sicher gestalten kann.

\section{Passwörter}\index{Passwörter} 
Die wohl verbreitetste Methode, die Menschen zur Authentifikation
benutzen sind Passwörter. Heutzutage begegnen uns Passwörter fast
überall.  Ob bei Twitter, Facebook, Youtube, in den eigenen
E-Mail-Konten, auf dem eigenen Computer, auf den Computern der
Universität, Amazon, Ebay, in einem Online-Shop oder andernorts, beinahe
überall werden Passwörter verwendet.

Wir modellieren dieses Szenario ganz allgemein: Ein Nutzer $U$ möchte
sich auf einem Server $S$ mittels Passwort $\pw$ einloggen.  Dabei
wünschen wir uns folgende Sicherheitseigenschaften:

\begin{itemize}
	\item Niemand außer $U$ kann sich bei $S$ als $U$ einloggen.
	\item Niemand soll das Passwort $\pw$ erfahren, nach Möglichkeit
auch nicht $S$.
\end{itemize}

Wir betrachten die zwei Angreifer Eve und Mallory. Eve kann die
Kommunikation zwischen $U$ und $S$ abhören, aber nicht
verändern. Mallory hat keinen Zugriff auf diese Kommunikation, ist dafür
jedoch in der Lage, die auf dem Server gespeicherte Benutzerdatenbank zu
erlangen, z.B. in dem er den Server hackt.\footnote{% Es ist übrigens
durchaus keine Seltenheit, dass Hacker Benutzerdatenbanken von gehackten
Webseiten öffentlich ins Internet stellen.  Dies ist besonders dann
gefährlich, wenn Benutzer ihre Passwörter bei anderen Diensten
wiederverwenden.  Noch schlimmer wird es, wenn das Passwort für einen
Benutzeraccount mit der (üblicherweise in Benutzerdatenbanken ebenfalls
hinterlegten) E-Mail-Adresse geteilt wird.  Denn ein solches
E-Mail-Konto kann leicht zum Generalschlüssel zu den Benutzeraccounts
des Opfers bei vielen anderen Webseiten werden.  Dafür muss der
Angreifer nur die "`Passwort Vergessen"'-Funktion auf diesen Webseiten
nutzen.  Häufig erhält der Nutzer dann entweder ein neues Passwort
zugesendet oder erhält eine Möglichkeit, selbst ein neues Passwort zu
wählen.  Hat der Angreifer aber Zugriff auf den E-Mail-Account des
Opfers, so kann er diese Funktion selbst nutzen und sich mit den neuen
Passwörtern auch bei anderen Internetseiten als das Opfer anmelden.  }
Wir betrachten diese Angreifer getrennt, d.h. Eve und Mallory
kooperieren nicht. Sollten sich Eve und Mallory doch zusammentun, so
können sie zusammen mindestens das erreichen, was zuvor schon einer
allein erreichen konnte.

Im einfachsten Verfahren verfügen sowohl $U$ als auch $S$ über das
Passwort $\pw$. Die Authentifikation geschieht, indem $U$ $S$ das
Passwort im Klartext übersendet. Dieses Verfahren ist in Abbildung
\ref{fig:auth:simplepassword} dargestellt.

\begin{figure}[h]
  \begin{center}
    \setlength{\unitlength}{1mm}
    \begin{picture}
      (50,10)(-5,-3) \put(0,-1.25){$U_\pw$}
      \put(45,-1.25){$S_\pw$} \put(8,0){\vector(1,0){36}} \put(23,2.5){$\pw$}
    \end{picture}
  \end{center}
  \caption{Einfache Benutzerauthentifikation mit Passwort.}
  \label{fig:auth:simplepassword}
\end{figure}

Dieses Verfahren bietet jedoch noch keinerlei Sicherheit. Eve, die die
Kommunikation abhören kann, erfährt unmittelbar das Passwort. Auch
Mallory, der die auf $S$ gespeicherte Passwortliste einsehen kann,
erfährt hier das Passwort.

Eine einfache Verbesserung bieten kryptographische Hashfunktionen. Der
Server speichert dann einen Hashwert des Passworts anstatt des Passworts
im Klartext. Dies ist in Abbildung \ref{fig:auth:storedpasswordhash}
gezeigt.

\begin{figure}[h]
  \begin{center} 
    \setlength{\unitlength}{1mm}
    \begin{picture}(50,10)(-5,-3) \put(0,-1.25){$U_\pw$}
      \put(45,-1.25){$S_{H(\pw)}$} \put(8,0){\vector(1,0){36}}
      \put(23,2.5){$\pw$}
    \end{picture}
  \end{center}
  \caption{Einfache Benutzerauthentifikation mit gespeichertem
    Passworthash.}
  \label{fig:auth:storedpasswordhash}
\end{figure}

Der Server kann nun immer noch überprüfen, ob das gesendete Passwort
$\pw$ mit dem gespeicherten Passwort übereinstimmt, indem er den
Hashwert des gesendeten Passworts mit dem gespeicherten Hashwert
vergleicht. Wegen der Kollisionsresistenz der Hashfunktion werden
unterschiedliche Passwörter zu unterschiedlichen Hashwerten führen. Wird
jedoch das richtige Passwort verwendet, so stimmen die Hashwerte
überein.

In diesem Verfahren kann Eve zwar immer noch das Passwort erhalten und
sich damit später als Benutzer $U$ bei $S$ anmelden.  Mallory jedoch,
der nur auf die Benutzerdatenbank von $S$ zugreifen kann, gelangt nur in
Besitz des Passworthashes $H(\pw)$, nicht jedoch von $\pw$ selbst.
Mallory kann sich also gegenüber $S$ nicht als der Benutzer $U$
ausgeben.\footnote{Ist Mallory jedoch ein besonders gewiefter Hacker und
hat Kontrolle über $S$, so könnte er jedoch auch eine Zeit lang alle an
den Server gesendeten Passwörter aufzeichnen und so an eine große Zahl
von Passwörtern gelangen. Meldet sich Benutzer $U$ in dieser Zeit bei
$S$ an, so gelangt Mallory ebenfalls an das Passwort $\pw$.}

In einer weiteren Variante sendet $U$ nicht sein Passwort im Klartext an
$S$, sondern hasht $\pw$ selbst und sendet diesen Hashwert an $S$. Dies
ist in Abbildung \ref{fig:auth:simplehashedpassword} dargestellt.

\begin{figure}[h]
  \begin{center} \setlength{\unitlength}{1mm}
    \begin{picture}(50,10)(-5,-3) \put(0,-1.25){$U_\pw$}
      \put(45,-1.25){$S_{H(\pw)}$} \put(8,0){\vector(1,0){36}}
      \put(21,2.5){$H(\pw)$}
    \end{picture}
  \end{center}
  \caption{Einfache Benutzerauthentifikation mit Hashfunktion und
    Passwort.}
  \label{fig:auth:simplehashedpassword}
\end{figure}

In dieser Variante erfährt Eve zwar nur den Hashwert $H(\pw)$ des
Passworts, dies reicht ihr jedoch, um sich später gegenüber $S$ als $U$
auszugeben. Auch Mallory, der $H(\pw)$ kennt, kann sich später als $U$
bei $S$ anmelden. Dafür erfahren jedoch weder Eve noch Mallory das
tatsächliche Passwort $\pw$. Dies schränkt die Wahrscheinlichkeit, dass
sich Eve oder Mallory bei einem anderen Server anmelden können, bei dem
$U$ das selbe Passwort verwendet, ein.

Die Sicherheitseigenschaften dieser drei einfachen Protokolle sind in
Tabelle \ref{table:auth:overview} zusammengefasst.\\
\begin{table}[h]
  \begin{center}
    \begin{tabular}{l||c|c|c|c} 
      & \multicolumn{2}{c|}{Eve} & \multicolumn{2}{|c}{Mallory}\\ 
      & lernt $\pw$ & Anmelden als $U$ & lernt $\pw$ & Anmelden als $U$
      \\\hline \hline 
      Verfahren 1(Abb. \ref{fig:auth:simplepassword}) & X & X & X & X \\\hline
      Verfahren 2 (Abb. \ref{fig:auth:storedpasswordhash}) & X & X & & \\\hline
      Verfahren 3 (Abb. \ref{fig:auth:simplehashedpassword}) & & X & & X \\
    \end{tabular}
  \end{center}
  \caption{Übersicht über die Sicherheitseigenschaften der drei
    betrachteten Protokolle. Die Spalten "`lernt $\pw$"' geben an, ob der
    jeweilige Angreifer das Passwort $\pw$ direkt lernt. Die Spalten
    "`Anmelden als $U$"' geben an, ob sich der jeweilige Angreifer gegenüber
    $S$ als $U$ ausgeben kann.}
  \label{table:auth:overview}
\end{table}

Man sieht, dass das dritte Verfahren zwar das Passwort $\pw$ besser
schützt als das zweite Verfahren. Dafür eröffnet es Mallory jedoch
wieder die Möglichkeit, sich bei $S$ als $U$ auszugeben.

\section{Wörterbuchangriffe}\index{Wörterbuchangriffe}
Wir betrachten nun noch einmal genauer die Möglichkeiten, aus $H(\pw)$
das benutzte Passwort $\pw$ zu rekonstruieren.

Wegen der Einwegeigenschaft von $H(\pw)$ ist es im Allgemeinen
schwierig, $\pw$ durch "`rückrechnen"' von $H$ zu erhalten. Die
Einwegeigenschaft von $H$ garantiert sogar, dass es sehr schwierig ist,
das Passwort $\pw$ zu finden, wenn das Passwort gleichverteilt zufällig
gewählt wurde.

Unglücklicherweise sind Passwörter jedoch meist alles Andere als
gleichverteilt zufällige Bitstrings.\footnote{ Eine Suche im Internet
fördert verschiedene Listen der am häufigsten benutzten Passwörter
zutage, darunter "`123456"', "`qwertz"' (im englischsprachigen Raum auch
"`qwerty"'), "`password"', oder "`abc123"'; außerdem findet man auch
Programme, die unter Nutzung solcher Listen versuchen, Urbilder zu einer
Liste von Hashes zu finden. Es gibt jedoch auch zahlreiche Anleitungen,
wie gute Passworte erstellt werden können.  }

Natürliche Sprachen wie deutsch oder englisch verfügen nur über wenige
tausend bis zehntausend Worte.  Wird ein solches natürlichsprachliches
Wort als Passwort verwendet, ist es ausreichend alle Worte dieser
Sprache zu hashen und die Hashwerte mit $H(\pw)$ zu vergleichen. Stimmt
der Hashwert eines natürlichen Wortes mit dem bekannten Hashwert
$H(\pw)$ überein, so hat man $\pw$ gefunden.  Es ist also
offensichtlich, dass natürlichsprachliche Worte keine guten Passwörter
sind.

Auch das Verwenden von gebräuchlichen Namen bringt keine wesentliche
Verbesserung, da es auch von diesen nur wenige tausend gibt. Auch das
Anhängen von Ziffern, Geburtstagen oder -jahren ergibt nicht genug
Kombinationsmöglichkeiten, um eine vollständige Suche ausreichend zu
erschweren.

\section{Brute-Force-Angriffe}

Solange also der Vorrat an Passwörtern klein genug ist, ist es mit
relativ wenig Aufwand möglich, zu gegebenem $H(\pw)$ das ursprüngliche
Passwort $\pw$ zu rekonstruieren. Deshalb konzentrieren wir uns nun auf
den Fall, wenn der Vorrat an Passworten sehr groß ist.

In diesem Fall ist es sehr aufwendig, für jeden zu brechenden Hashwert
$H(\pw)$ alle möglichen Passworte durchzuprobieren. Gibt es insgesamt
$N$ Passwörter, dann muss man $H$ etwa $\mathcal{O}(N)$ mal
auswerten. Es ist daher (aus Angreifersicht) wünschenswert, eine
vollständige Liste aller möglichen Passworte $\pw$ und ihrer Hashwerte
$H(\pw)$ zu besitzen. Dies ist in Abbildung \ref{fig:auth:listofhashes}
illustriert.

\begin{figure}[h]
  \begin{align*}
    (\quad H(\pw_1) \qquad&,\qquad \pw_1 \quad) \\
    (\quad H(\pw_2) \qquad&,\qquad \pw_2 \quad) \\ 
    \vdots \qquad&\qquad \vdots
  \end{align*}
  \caption{Eine Liste aller Passwörter und ihrer Hashwerte.}
  \label{fig:auth:listofhashes}
\end{figure}

Ist diese Liste nach $H(\pw)$ sortiert, kann man zu einem gegebenen
Hashwert sogar durch binäre Suche sehr effizient das zugrundeliegende
Passwort bestimmen, man braucht dazu nur $\mathcal{O}(\log_2 N)$
Operationen. Für sehr große Mengen an möglichen Passwörtern werden
jedoch auch diese Listen sehr groß ($\Omega(N)$), und es entsteht ein
Speicherplatzproblem.

\section{Kompression von Hashtabellen/Time Memory Tradeoff}
\index{Hashtabelle}\index{Time-Memory-Tradeoff}
Einen Mittelweg zwischen sehr großer Suchzeit (ohne vorberechnete
Tabelle aller Passworte und Hashwerte) und sehr viel
Speicherplatzverbrauch (mit vollständiger Liste aller Passworte und
ihrer Hashwerte) liefert die Kompression von Hashtabellen. Man
bezeichnet diese Technik auch als "`Time Memory Tradeoff"'.

Unglücklicherweise sind gute, kryptographische Hashwerte quasi zufällig
und nur sehr schwer zu komprimieren. Daher können gängige
Kompressionsverfahren nicht angewendet werden.

Tatsächlich ergibt sich jedoch ein sehr einfaches, maßgeschneidertes
Kompressionsverfahren für solche Hashtabellen, dass sogar eine sehr
effiziente Suche erlaubt.  Hierzu betrachtet man Hash\emph{ketten}.
Eine \emph{Hashkette}\index{Hashkette} (vgl. Abbildung
\ref{fig:auth:hashchain}) beginnt mit einem Passwort $\pw_1$ aus dem
Vorrat aller Passwörter.

Anschließend wird dieses Passwort gehasht, um $H(\pw_1)$ zu erhalten.
Nun wird eine sogenannte \emph{Reduktionsfunktion} $f$ benutzt, um
diesen Hashwert auf ein neues Passwort $\pw_2 = f(H(\pw_1))$ aus dem
Passwortraum abzubilden.  Anschließend wird dieses wieder zu $H(\pw_2)$
gehasht.  Dieser Hashwert wird erneut durch $f$ auf ein Passwort $\pw_3$
abgebildet, usw.  Dieser Prozess kann theoretisch beliebig lange
fortgeführt werden. Man beschränkt dies jedoch auf eine frei wählbare
Anzahl von Iterationen $m$.

\newcommand{\RTF}[1][]{\mathit{f}\if!#1!\else_{#1}\fi}
\newcommand{\VRTF}[1][]{\xlongrightarrow{\RTF[#1]}}
\newcommand{\hasharrow}{\stackrel{H}{\longrightarrow}}
\begin{figure}[h]
  \begin{equation*} \pw_1 \hasharrow H(\pw_1) \VRTF \pw_2
    \hasharrow H(\pw_2) \VRTF %\pw_3 \quad \ldots \quad \hasharrow
    H(\pw_{m-1}) \VRTF \pw_m \hasharrow H(\pw_m)
  \end{equation*}
  \caption{Eine Hashkette.}
  \label{fig:auth:hashchain}
\end{figure}

Eine solche Kette stellen wir auch wie in Abbildung
\ref{fig:auth:hashchainalternative} dar.

\begin{figure}[h]
  \begin{equation*} (H(\pw_{1}),\pw_{1}) \VRTF
    (H(\pw_{2}),\pw_{2}) \VRTF\quad \ldots \quad \VRTF (H(\pw_{m}),\pw_{m})
  \end{equation*}
  \caption{Eine alternative Darstellung für Hashketten.}
  \label{fig:auth:hashchainalternative}
\end{figure}

Es ist leicht einzusehen, dass zur Konstruktion einer solchen Hashkette
nur das Passwort $\pw_1$ benötigt wird. Man kann $\pw_1$ also als stark
komprimierte Form der Hashkette verstehen, da man die gesamte Kette aus
$\pw_1$ berechnen kann.

Anstelle einer vollständigen Liste aller möglichen Passwörter speichert
man nun eine Menge von $n$ Hashketten. Diese kann man tabellarisch wie
in Abbildung \ref{fig:auth:hashchains} darstellen.

\begin{figure}[h]
  \begin{gather*} (H(\pw_{1,1}),\pw_{1,1}) \VRTF
    (H(\pw_{1,2}),\pw_{1,2}) \VRTF\ldots\VRTF (H(\pw_{1,m}),\pw_{1,m})\\
    (H(\pw_{2,1}),\pw_{2,1}) \VRTF (H(\pw_{2,2}),\pw_{2,2}) \VRTF\ldots\VRTF
    (H(\pw_{2,m}),\pw_{2,m})\\ 
    \vdots\\ 
    (H(\pw_{n,1}),\pw_{n,1}) \VRTF
    (H(\pw_{n,2}),\pw_{n,2}) \VRTF\ldots\VRTF (H(\pw_{n,m}),\pw_{n,m})
  \end{gather*}
  \caption{Tabellarische Darstellung von $n$ Hashketten der Länge
    $m$.}
  \label{fig:auth:hashchains}
\end{figure}

Hierbei nimmt man in Kauf, dass möglicherweise nicht alle Passwörter in
der so entstehenden Tabelle auftauchen. Den Anteil dieser Passwörter
kann man jedoch verringern, in dem man die Anzahl der Hashketten $n$
oder die Länge der Hashketten $m$ erhöht.

Um nun eine Kompression der Tabelle bei gleichzeitiger effizienter Suche
zu erreichen, speichert man für jede Hashkette $i$ nur das erste
Passwort $\pw_{i,1}$ und den letzten Hashwert $H(\pw_{i,m})$. Wenn $m
\cdot n$ ungefähr der Zahl aller Passwörter entspricht, dann ist die so
entstehene Tabelle ungefähr um den Faktor $m$ kleiner als eine
vollständige Auflistung aller Passwörter und ihrer Hashwerte.

Tabelle \ref{table:auth:timememorytradeofffinal} zeigt diese "`komprimierte"' Form.
\begin{table}[!h]
  \begin{equation*}
    \begin{array}{|c|c|} \hline \pw_{1,1} & H(\pw_{1,m})\\
      \hline \pw_{2,1} & H(\pw_{2,m})\\
      \hline
      \vdots & \vdots \\
      \hline
      \pw_{n,1} & H(\pw_{n,m})\\
      \hline
    \end{array}
  \end{equation*}
  \caption{Die komprimierte Hashtabelle.}
  \label{table:auth:timememorytradeofffinal}
\end{table}

Diese Tabelle wird nun nach der Spalte der Hashwerte $H(\pw_{i,m})$
sortiert, um eine effiziente Suche nach Hashwerten zu ermöglichen.

Nun sei $H(\pw^*)$ der dem Angreifer bekannte Passworthash. Das Ziel des
Angreifers ist es, mittels der oben gezeigten Tabelle das Passwort
$\pw^*$ zu rekonstruieren.

Zunächst nimmt der Angreifer an, dass das gesuchte Passwort $\pw^*$ als
letztes Passwort in einer der Hashketten auftaucht.  Es soll also $\pw^*
= \pw_{i,m}$ für ein $i$ gelten.  Wenn diese Annahme zutrifft, dann ist
$H(\pw^*)$ also $H(\pw_{i,m})$. Deshalb sucht der Angreifer in der
zweiten Spalte von Tabelle \ref{table:auth:timememorytradeofffinal} nach
$H(\pw^*)$. Dies ist effizient mittels binärer Suche möglich.  War die
Hypothese korrekt, dann liefert diese Suche einen Treffer in der $i$-ten
Zeile. Dann kann der Angreifer die Hashkette von $\pw_{i,1}$ ausgehend
rekonstruieren und erhält so $\pw_{i,m}$. Dies ist das gesuchte Passwort
$\pw^*$. War die Hypothese falsch, dann liefert diese binäre Suche
keinen Treffer.

In diesem Fall stellt der Angreifer eine neue Hypothese auf: "`Das
gesuchte Passwort $\pw^*$ ist als zweitletztes Passwort in einer der
Hashketten enthalten."'  Dann gilt also $\pw^* = \pw_{i,m-1}$ für ein
$i$, und daher auch $H(\pw_{i,m}) = H(f(H(\pw^*)))$, denn $\pw_{i,m}$
ist genau $f(H(\pw_{i,m-1}))$.  Um zu überprüfen, ob diese Hypothese
stimmt, berechnet der Angreifer daher den Hash $H(f(H(\pw^*)))$ und sucht in
Tabelle \ref{table:auth:timememorytradeofffinal} nach dem Ergebnis
dieser Berechnung.  Liefert die Suche einen Treffer in Kette $i$, kann
der Angreifer diese Kette wieder von $\pw_{i,1}$ neu aufbauen und
erfährt so $\pw_{i,m-1} = \pw^*$.  Liefert die Suche keinen Treffer,
dann war die Hypothese falsch, und der Angreifer fährt mit der nächsten
Hypothese fort: das gesuchte Passwort soll als drittletztes in einer der
Hashketten zu finden sein.  Diese Hypothese testet der Angreifer durch
eine Suche nach $H(f(H(f(H(\pw^*)))))$, usw.

Nacheinander testet der Angreifer so alle Positionen in den
Hashketten. Liefert eine der Suchen einen Treffer, so hat der Angreifer
das Passwort gefunden. Andernfalls ist das gesuchte Passwort nicht in
der Hashtabelle enthalten.

\begin{beispiel}
\label{ex:auth:timememorytradeoff} Wir betrachten als Raum aller
möglichen Passwörter die Buchstaben "`a"' bis "`z"'.  Die angewendete
Hashfunktion sei die schon bereits erwähnte SHA-1-Funktion.  Um einen
Hashwert zurück auf ein Passwort abzubilden, interpretieren wir den
Hashwert als natürliche Zahl $h$ in Hexadezimal-Darstellung, und
berechnen $h \mod 26$.  Die so entstehenden Zahlen von $0$ bis $25$
bilden wir auf natürliche Weise zurück auf die Buchstaben "`a"' bis
"`z"' ab.  Wir wählen $m = 4$ als Kettenlänge.  Da es insgesamt 26
mögliche Passwörter gibt, könnte $n = 7$ ausreichen, damit alle
Passworte an irgendeiner Stelle der Hashketten vorkommen, denn insgesamt
gibt es $7 \cdot 4 = 28$ Passwörter in den Hashketten.  Wir wollen es
mit $n = 7$ Hashketten versuchen.

Als Startpassworte der Hashketten wählen wir die Buchstaben "`a"' bis
"`g"'. Die so entstehenden Hashketten sind in Abbildung
\ref{fig:auth:timememorytradeoff:hashchains} gezeigt.

\begin{figure}[h!] 
%complete results: % a =>
%86f7e437faa5a7fce15d1ddcb9eaeaea377667b8 => o \\% b =>
%e9d71f5ee7c92d6dc9e92ffdad17b8bd49418f98 => i \\% c =>
%84a516841ba77a5b4648de2cd0dfcb30ea46dbb4 => w \\% d =>
%3c363836cf4e16666669a25da280a1865c2d2874 => c \\% e =>
%58e6b3a414a1e090dfc6029add0f3555ccba127f => v \\% f =>
%4a0a19218e082a343a1b17e5333409af9d98f0f5 => l \\% g =>
%54fd1711209fb1c0781092374132c66e79e2241b => n \\% h =>
%27d5482eebd075de44389774fce28c69f45c8a75 => d \\% i =>
%042dc4512fa3d391c5170cf3aa61e6a638f84342 => u \\% j =>
%5c2dd944dde9e08881bef0894fe7b22a5c9c4b06 => o \\% k =>
%13fbd79c3d390e5d6585a21e11ff5ec1970cff0c => o \\% l =>
%07c342be6e560e7f43842e2e21b774e61d85f047 => l \\% m =>
%6b0d31c0d563223024da45691584643ac78c96e8 => k \\% n =>
%d1854cae891ec7b29161ccaf79a24b00c274bdaa => w \\% o =>
%7a81af3e591ac713f81ea1efe93dcf36157d8376 => w \\% p =>
%516b9783fca517eecbd1d064da2d165310b19759 => h \\% q =>
%22ea1c649c82946aa6e479e1ffd321e4a318b1b0 => i \\% r =>
%4dc7c9ec434ed06502767136789763ec11d2c4b7 => b \\% s =>
%a0f1490a20d0211c997b44bc357e1972deab8ae3 => d \\% t =>
%8efd86fb78a56a5145ed7739dcb00c78581c5375 => b \\% u =>
%51e69892ab49df85c6230ccc57f8e1d1606caccc => g \\% v =>
%7a38d8cbd20d9932ba948efaa364bb62651d5ad4 => w \\% w =>
%aff024fe4ab0fece4091de044c58c9ae4233383a => q \\% x =>
%11f6ad8ec52a2984abaafd7c3b516503785c2072 => s \\% y =>
%95cb0bfd2977c761298d9624e4b4d4c72a39974a => q \\% z =>
%395df8f7c51f007019cb30201c49e884b46b92fa => m \\
\begin{center}
  \begin{tabular}{c@{$\;\hasharrow\;$}c@{$\;\VRTF\;$}c@{$\;\hasharrow\;$}c@{$\;\VRTF\;$}c@{$\;\hasharrow\;$}c@{$\;\VRTF\;$}c@{$\;\hasharrow\;$}c}
    a & 86f7\ldots & o & 7a81\ldots & w & aff0\ldots & q & 22ea\ldots\\
    b &e9d7\ldots & i & 042d\ldots & u & 51e6\ldots & g & 54fd\ldots\\
    c &84a5\ldots & w & aff0\ldots & q & 22ea\ldots & i & 042d\ldots\\
    d &3c36\ldots & c & 84a5\ldots & w & aff0\ldots & q & 22ea\ldots\\ 
    e &58e6\ldots & v & 7a38\ldots & w & aff0\ldots & q & 22ea\ldots\\
    f &4a0a\ldots & l & 07c3\ldots & l & 07c3\ldots & l & 07c3\ldots\\
    g &54fd\ldots & n & d185\ldots & w & aff0\ldots & q & 22ea\ldots\\
  \end{tabular}
  \caption{Die in Beispiel \ref{ex:auth:timememorytradeoff}
    erzeugten Hashketten. Aus Platzgründen sind die Hashwerte auf die ersten
    vier Hexadezimalstellen gekürzt.}
  \label{fig:auth:timememorytradeoff:hashchains}
\end{center}
\end{figure}

Gespeichert werden von diesen Hashketten nur die Startpassworte sowie
die letzten Hashwerte. Die Tabelle wird nach den Hashwerten
sortiert. Das Ergebnis ist in Tabelle
\ref{table:auth:timememorytradeoff:hashtable} zu sehen.

\begin{table}[h!]
  \begin{center}
    \begin{tabular}{|c|c|} \hline 
      c & 042d\ldots\\ \hline 
      f & 07c3\ldots\\ \hline 
      a & 22ea\ldots\\ \hline 
      d & 22ea\ldots\\ \hline 
      e & 22ea\ldots\\ \hline 
      g & 22ea\ldots\\ \hline 
      b & 54fd\ldots\\ \hline
    \end{tabular}
  \end{center}
  \caption{Die komprimierte Hashtabelle aus Beispiel
    \ref{ex:auth:timememorytradeoff}.}
  \label{table:auth:timememorytradeoff:hashtable}
\end{table}

Der dem Angreifer bekannte Hashwert sei nun $H(\pw^*) = 042d\cdots$.
Der Angreifer stellt nun zunächst die Hypothese auf, dass das gesuchte
Passwort $\pw^*$ als letztes in einer der Hashketten auftaucht. Er sucht
deshalb in Tabelle \ref{table:auth:timememorytradeoff:hashtable} nach
dem ihm bekannten Hashwert $042d\cdots$. Diese Suche liefert einen
Treffer in Zeile $i = 1$. Die Hypothese war also korrekt. Nun weiß der
Angreifer, dass das gesuchte Passwort $\pw^* = \pw_{1,m}$ ist. Er
rekonstruiert also die Hashkette ausgehend vom Startpasswort "`c"' und
erhält so das gesuchte Passwort "`i"'.
\end{beispiel}

\begin{beispiel} Wir betrachten wieder die komprimierte Hashtabelle aus
dem vorherigen Beispiel. Diesmal sei der dem Angreifer bekannte Hashwert
aber $H(\pw^*) = 51e6\ldots$.  Der Angreifer möchte nun testen, ob das
gesuchte Passwort als letztes in einer der Hashketten aus Abbildung
\ref{fig:auth:timememorytradeoff:hashchains} auftritt.  Doch der
gesuchte Hashwert $51e6\ldots$ taucht nicht in der zweiten Spalte von
Tabelle \ref{table:auth:timememorytradeoff:hashtable} auf.  Daher war
diese erste Hypothese falsch. Der Angreifer berechnet $f(H(\pw^*)) = g$
und $H(f(H(\pw^*))) = 54fd\ldots$.  Eine Suche nach diesem Hashwert
liefert tatsächlich einen Treffer in Zeile $i = 7$ der Tabelle
\ref{table:auth:timememorytradeoff:hashtable}.  Der Angreifer
rekonstruiert also die Hashkette ausgehend vom Buchstaben $b$ und erhält
$\pw_{i,m-1} = \pw_{i,3} = u$.  Dies ist das gesuchte Passwort
$\pw^*$.
\end{beispiel}

\begin{beispiel} Wir betrachten wieder die selbe Hashtabelle, diesmal
sei der gesuchte Hashwert jedoch $H(\pw^*) = 7a38\ldots$.  Dieser kommt
in der zweiten Spalte von Tabelle
\ref{table:auth:timememorytradeoff:hashtable} nicht vor, also taucht
$\pw^*$ nicht an der letzten Stelle einer Hashtabelle auf.  Der
Angreifer berechnet daraufhin $f(H(\pw^*)) = w$ und $H(f(H(\pw^*))) =
aff0\ldots$. Auch dieser Hashwert taucht nicht in Tabelle
\ref{table:auth:timememorytradeoff:hashtable} auf, daher ist das
gesuchte Passwort auch nicht als zweitletztes Passwort in einer der
Hashketten enthalten.  Deshalb setzt der Angreifer die Berechnung fort:
er erhält $f(H(f(H(\pw^*)))) = q$ und $H(f(H(f(H(\pw^*))))) =
22ea\ldots$. Dieser Wert taucht gleich vier Mal in Tabelle
\ref{table:auth:timememorytradeoff:hashtable} auf. Der Angreifer
rekonstruiert daher die vier Ketten ausgehend von "`a"', "`d"', "`e"'
und "`g"', und findet schließlich in der von "`e"' ausgehenden Hashkette
das gesuchte Passwort "`v"'.
\end{beispiel}

Dieses Beispiel illustriert bereits eines der Probleme solcher
Hashtabellen: Es kann passieren, dass mehrere Hashketten, die mit
verschiedenen Passworten beginnen, "`zusammenlaufen"'.  Dies kann
passieren, wenn $f$ eine Kollision liefert, also verschiedene Hashwerte
auf das selbe Passwort abbildet. (Dies lässt sich nur schwer vermeiden,
da es im Allgemeinen wesentlich mehr Hashwerte als Passworte gibt.)
Tritt ein solcher Fall auf, laufen die Hashketten ab diesem Punkt auch
identisch weiter.

Im obigen Beispiel ist z.B. $f(H(o)) = f(H(c)) = f(H(v)) = f(H(n)) =
w$. Deshalb laufen in Abbildung
\ref{fig:auth:timememorytradeoff:hashchains} die Hashketten "`a"',
"`d"', "`e"' und "`g"' zusammen, und enden schließlich gemeinsam auf
$H(q) = 22ea\cdots$. Tatsächlich tauchen die Passwörter "`w"' und "`q"'
sogar noch in Hashkette "`c"' auf. Dort befinden sie sich jedoch weiter
vorne, deshalb endet diese Kette nicht auf $H(q) = 22ea\cdots$ sondern
auf $H(i)$.

Dies führt einerseits dazu, dass gewisse Passworte mehrfach in der
Hashtabelle vorkommen. Dies ist aus Angreifersicht noch kein
Problem. Andererseits nehmen diese mehrfach vorkommenden Passworte
jedoch auch Platz für andere Passwörter weg.

\begin{beispiel} 
  Wir betrachten wieder die obigen Hashtabellen, dieses
  Mal ist der gesuchte Hashwert $H(\pw^*) = 95cb\ldots$.
  
  Dieser Hashwert taucht jedoch nicht in Tabelle
  \ref{table:auth:timememorytradeoff:hashtable} auf, daher ist das
  gesuchte Passwort nicht an letzter Stelle einer der Hashketten.
  
  Anschließend sucht der Angreifer nach $H(f(H(\pw^*))) =
  22ea\ldots$. Diese Suche liefert vier Treffer, in den Hashketten "`a"',
  "`d"', "`e"' und "`g"'. Der Angreifer rekonstruiert also diese
  Hashketten bis zur zweitletzten Position, und findet den Buchstaben $w$
  an allen Stellen. Es gilt aber $H(w) = aff0\ldots \neq 95cb\ldots =
  H(\pw^*)$. Dieses Passwort ist also \emph{nicht} korrekt.
  
  Der Angreifer setzt die Suche fort und berechnet
  \[H(f(H(f(H(\pw^*))))) = 042d\ldots.\] Dies liefert wieder einen Treffer
  in Hashkette "`c"', aber auch diesmal liefert die Rekonstruktion der
  Hashkette wieder das falsche Passwort "`w"'.
  
  Deshalb fährt der Angreifer weiter fort und berechnet
  \[H(f(H(f(H(f(H(\pw^*))))))) = 51e6\ldots.\] Dies liefert keinen Treffer.
  
  Nun hat der Angreifer alle möglichen Hypothesen getestet: Dass
  das Passwort als letztes (viertes), zweitletztes (drittes), drittletztes
  (zweites) oder viertletztes (erstes) in einer der Hashketten vorkommt.
  All diese Hypothesen waren falsch, also ist das gesuchte Passwort nicht
  in der Hashtabelle enthalten.  (Das gesuchte Passwort war "`y"'.)
\end{beispiel}
      
Dieses Beispiel illustriert noch ein weiteres Problem von Kollisionen:
Diese können zu falsch-positiven Treffern führen. Dieses Problem lässt
sich jedoch leicht beheben, in dem man jeden gefunden
Passwort-Kandidaten hasht und den so entstehenden Hashwert mit dem
vorgegebenen Hashwert $H(\pw^*)$ vergleicht.

Von den insgesamt 26 möglichen Passwörtern lassen sich mit Hilfe der
Tabelle \ref{table:auth:timememorytradeoff:hashtable} 15 Passwörter
rekonstruieren. Die Tabelle überdeckt also nur etwa 58\% des
Passwortraums.

Es sei wieder $N$ die Zahl aller Passwörter. Zum Speichern der
komprimierten Tabelle \ref{table:auth:timememorytradeofffinal} braucht
man etwa $\Omega(n)$ Speicherplatz.\footnote{Zur Berechnung aller
Hashketten benötigt man aber $\Omega(m \cdot n) \approx N$
Operationen. Anschließend müssen diese Hashketten noch sortiert werden.}
Ist $m \cdot n \approx N$, so schrumpft der Platzbedarf gegenüber einer
vollständigen Tabelle aller Passwörter und ihrer Hashwerte also etwa um
den Faktor $m$.

Um nach einem Hashwert zu suchen, benötigt man hier $\mathcal{O}(m \cdot
\log_2(n))$ Operationen, während man bei einer vollständigen Tabelle nur
$\mathcal{O}(\log_2(N))$ Operationen benötigt. Der Zeitbedarf zur Suche
nach einem Passwort wächst also etwa um einen Faktor von $m \cdot
\frac{\log_2(n)}{\log_2(N)}$.

\section{Rainbow Tables}
\index{Rainbow-Table}
Eine Technik das Zusammenlaufen von Ketten zumindest partiell zu
verhindern sind sogenannte Rainbow Tables. Dabei verwendet man nicht
\emph{eine} Reduktionsfunktion, sondern $m-1$ verschiedene
Reduktionsfunktionen $f_i$, wobei jede Reduktionsfunktion $f_i$ für die
$i$-te Reduktion in einer Hashkette verwendet wird. Abbildung
\ref{fig:auth:rainbowhashchain} zeigt eine solche Hashkette.

\begin{figure}[h]
  \begin{equation*}
    \pw_1 \hasharrow H(\pw_1)
    \VRTF[1] \pw_2 \hasharrow H(\pw_2)
    \VRTF[2] %\pw_3
    \quad \ldots \quad 
    \hasharrow H(\pw_{m-1})
    \VRTF[m-1] \pw_m \hasharrow H(\pw_m)
  \end{equation*}
  \caption{Eine Hashkette mit verschiedenen Reduktionsfunktionen.}
  \label{fig:auth:rainbowhashchain}
\end{figure}

Diese Änderung verhindert, dass Hashketten zusammenlaufen, solange die
Kollision an verschiedenen Stellen in den Hashketten auftreten.

\begin{beispiel}
\label{ex:auth:rainbowtables} Wir wollen dies wieder mit Hilfe von
Beispiel \ref{ex:auth:timememorytradeoff} verdeutlichen. Wir definieren
dazu die Reduktionsfunktionen $f_i$, die jeden Hashwert wieder als
natürliche Zahl $h$ in Hexadezimaldarstellung interpretieren. Zu dieser
Zahl wird dann $i$ addiert, und das Ergebnis modulo 26 reduziert. Es ist
also $f_i(h) = h + i \mod 26$. Dies führt zu den in Abbildung
\ref{fig:auth:timememorytradeoff:rainbowhashchains} gezeigten
Hashketten.

\begin{figure}[h!]
  \begin{center}
    \begin{tabular}{c@{$\;\hasharrow\;$}c@{\ldots$\;\VRTF[1]\;$}c@{$\;\hasharrow\;$}c@{\ldots$\;\VRTF[2]\;$}c@{$\;\hasharrow\;$}c@{\ldots$\;\VRTF[3]\;$}c@{$\;\hasharrow\;$}c@{\ldots}}
      a & 86f7 & p & 516b & j & 5c2d & r & 4dc7\\
      b & e9d7 & j & 5c2d & q & 22ea & l & 07c3\\ 
      c & 84a5 & x & 11f6 & u & 51e6 & j & 5c2d\\
      d & 3c36 & d & 3c36 & e & 58e6 & y & 95cb\\
      e & 58e6 & w & aff0 & s & a0f1 & g & 54fd\\
      f & 4a0a & k & 13fb & q & 22ea & l & 07c3\\ 
      g & 54fd & m & 6b0d & m & 6b0d & n & d185\\
    \end{tabular}
    \caption{Die in Beispiel \ref{ex:auth:rainbowtables} erzeugten
      Hashketten.}
    \label{fig:auth:timememorytradeoff:rainbowhashchains}
\end{center}
\end{figure} 
Man sieht, dass die Hashketten "`b"' und "`f"'
zusammenlaufen, da die Funktion $f_2$ eine Kollision liefert.
Andererseits laufen z.B. die Hashketten "`d"' und "`e"' \emph{nicht}
zusammen, obwohl beide ein $e$ enthalten. Denn hier liegen die "`e"'s an
\emph{verschiedenen} Positionen, und die Hashwerte werden deshalb im
Anschluss von verschiedenen Reduktionsfunktionen auf unterschiedliche
Passworte abgebildet.

In Abbildung \ref{fig:auth:timememorytradeoff:hashchains} wurde noch der
Buchstabe "`l"' immer auf sich selbst abgebildet. Deswegen enthielt die
Kette "`f"' dort 3 "`l"'s nacheinander.  So etwas tritt hier nicht
auf. Zwar werden immer noch Buchstaben auf sich selbst abgebildet (siehe
z.B. das "`m"' in Kette "`g"'). Da jedoch immer verschiedene
Reduktionsfunktionen verwendet werden, wird das zweite "`m"' nicht mehr
auf sich selbst sondern auf "`n"' abgebildet.
\end{beispiel}

Wegen dieser Eigenschaft haben Rainbow Tables im Allgemeinen eine
bessere Abdeckung des Passwort-Raums als gleich große Hashtabellen mit
nur einer Reduktionsfunktion. Unsere Rainbow Table hier deckt z.B. 20
der 26 möglichen Passwörter ab, also ca. 77\% des Passwortraums. Die
Hashtabelle mit nur einer Reduktionsfunktion deckte nur 15 Passwörter
(58\%) ab. \footnote{Man kann auch vorberechnete Rainbow Tables für
wenige hundert Euro kaufen. Diese erreichen häufig Abdeckungsraten von
weit über 90\%, und werden wegen ihrer Größe gleich auf mehreren
externen Terabyte-Festplatten geliefert. % http://rainbowtables.org 
}

Der Begriff "`Rainbow Tables"' bezieht sich auf die verschiedenen
"`Farben"' der Reduktionsfunktionen $f_i$.

Die Suche in Rainbow Tables funktioniert konzeptionell genau wie bei
Hashtabellen mit nur einer Reduktionsfunktion: Man testet nacheinander
die Hypothesen "`Das gesuchte Passwort taucht als $j$-tes ein einer
Hashkette $i$ auf."' Um zu testen, ob das gesuchte Passwort $\pw^*$ an
Stelle $m$ ist, muss man also nach $H(\pw^*)$ suchen. Um zu testen, ob
das gesuchte Passwort an Stelle $m-1$ ist, berechnet man
$H(f_{m-1}(H(\pw^*)))$ und sucht nach diesem Hashwert in der Rainbow
Table. Um zu testen, ob $\pw^*$ an Stelle $m-2$ liegt, berechnet man
$H(f_{m-1}(H(f_{m-2}(H(\pw^*)))))$ und sucht nach diesem Ergebnis,
usw.

\begin{beispiel} 
  Wir verwenden die Hashketten aus Abbildung
  \ref{fig:auth:timememorytradeoff:rainbowhashchains}. Aus diesen ergibt
  sich die komprimierte Rainbow Table \ref{table:auth:rainbowtable}.
  
  \begin{table}[h!]
    \begin{center}
      \begin{tabular}{|c|c|} \hline 
        b & 07c3\ldots\\\hline 
        f & 07c3\ldots\\\hline 
        a & 4dc7\ldots\\\hline 
        e & 54fd\ldots\\\hline 
        c & 5c2d\ldots\\\hline 
        d & 95cb\ldots\\\hline 
        g & d185\ldots\\\hline
      \end{tabular}
    \end{center}
    \caption{Die komprimierte Rainbow Table für die Hashketten aus
      Abbildung \ref{fig:auth:timememorytradeoff:rainbowhashchains}.}
    \label{table:auth:rainbowtable}
  \end{table}

  Der gesuchte Hashwert sei $H(\pw^*) = 11f6\cdots$.
  Die Hypothese, dass $\pw^* = \pw_{i,m}$ für ein $i$ sei, stellt sich als
  falsch heraus, denn $H(\pw^*)$ taucht nicht in der m-ten Spalte der
  Rainbow Table auf, und somit auch nicht in Tabelle
  \ref{table:auth:rainbowtable}. 
  
  Man testet daher, ob $\pw^* = \pw_{i,m-1}$ für ein $i$ ist. Dazu
  berechnet man $f_{m-1}(H(\pw^*)) = v$ und $H(f_{m-1}(H(\pw^*))) =
  7a38\cdots$. Die binäre Suche nach diesem Wert liefert ebenfalls kein
  Ergebnis, daher war auch diese Hypothese falsch.
  
  Die nächste Hypothese ist, dass $\pw^* = \pw_{i,m-2}$ für ein
  $i$ sein soll.  Man berechnet 
  \begin{align*}
    f_{m-2}(H(\pw^*))               &= u\\
    H(f_{m-2}(H(\pw^*)))            &= 51e6\cdots\\
    f_{m-1}(H(f_{m-2}(H(\pw^*))))    &= j\\
    H(f_{m-1}(H(f_{m-2}(H(\pw^*))))) &= 5c2d\cdots  
  \end{align*}

  Diesmal liefert die Suche in Tabelle \ref{table:auth:rainbowtable}
  einen Treffer in Zeile $i = 5$. Das Startpasswort der Hashkette war
  "`c"'. Deshalb rekonstruiert man die Hashkette ausgehend von $c$ und
  findet $\pw^* = \pw_{5,2} = x$.
\end{beispiel}

Anders als bei Hashtabellen mit nur einer Reduktionsfunktion benötigt
man hier jedoch $\mathcal{O}(m^2 \cdot \log_2(n))$ Operationen für eine
Suche, da man für jede Hypothese die Berechnung des entsprechenden
Hashwerts neu beginnen muss.

\section{Gegenmaßnahmen}


Nachdem wir nun gesehen haben, wie man bekannte Passworthashes mit Hilfe
von vorberechneten Tabellen relativ effizient auf ihr Passwort zurück
abbilden kann, wollen wir uns nun noch einmal damit befassen, wie man
solche Angriffe erschwert.

Eine einfache Lösung bieten sogenannte "`gesalzene"' Hashwerte. In
diesem Szenario ist jedem Benutzer noch ein individuelles "`Salz"' $s$
(englisch "`salt"') zugeordnet. Der Hashwert des Passwortes ist dann
$H(s || \pw)$. In der Praxis ist dies oft ein zufälliger String, der vorn
oder hinten an das Passwort angehängt wird.

Vorberechnete Hash-Tabellen\index{Hashtabelle} (wie z.B. Rainbow Tables)
werden dadurch nutzlos. Die Erstellung von Rainbow Tables o.Ä. ist erst
dann sinnvoll, wenn der Angreifer den Salt kennt. Und selbst dann hilft
die Rainbow Table nur beim Knacken \emph{eines} Passworthashes, da
verschiedene Benutzer im Allgemeinen verschiedene Salts haben. Dann
liefert die Vorberechnung von Rainbow Tables aber auch keinen Vorteil
gegenüber dem Ausprobieren aller möglichen Passworte.


Theoretisch wäre es zwar auch möglich, eine Rainbow Table über
\emph{alle} Kombinationen von Salt und Passwort zu erstellen. Für
ausreichend lange und zufällige Salts ist der Aufwand hierfür jedoch
nicht praktikabel.

Eine zweite einfache Möglichkeit ist die Wiederholung der
Hashfunktion. Dies wird auch als "`Key Strengthening"' bezeichnet.

In diesem Fall ist der gespeicherte Passworthash nicht mehr $H(\pw)$,
sondern statt dessen $H(H(\ldots H(\pw)\ldots))$. Wiederholt man die Funktion $H$
z.B. $n$ mal, so wird der Aufwand, der zum Knacken von Passwörtern oder
zur Erstellung einer Rainbow Table benötigt wird, ver-$n$-facht.

Andererseits wird auch der Aufwand zur Verifikation eines Passworts um
den Faktor $n$ gesteigert, da der Server $S$ nun bei jeder versuchten
Anmeldung die Hashfunktion $H$ insgesamt $n$ mal ausführen muss.

Diese Methode kann jedoch trotzdem sinnvoll sein, da $S$ im Allgemeinen
weniger Passworthashes berechnen muss als ein Angreifer. Selbst bei
einem sehr viel genutzten Dienst sind höchstens wenige Milliarden
Login-Versuche pro Tag zu erwarten. Wiederholt man die Funktion $H$ 1000
mal, so muss $S$ etwa $10^{12}$ Auswertungen von $H$ pro Tag
durchführen. Ist der Passwortraum aber größer als $10^9$,
z.B. $10^{15}$, so müsste der Angreifer insgesamt $10^{18}$ mal die
Funktion $H$ auswerten. Dies stellt den Angreifer vor eine deutlich
größere Herausforderung als den Betreiber des Servers $S$.

%%% Local Variables:
%%% mode: latex
%%% TeX-master: "skript"
%%% End:
